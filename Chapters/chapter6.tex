\chapter{Conclusions and Future Work}  
\label{chapter6}

This chapter presents a discussion on the findings relating to the simulation of developnig a scalable, carbon-neutral ride-sharing service based on AEVs and SECs and outlines the overall contributions of this research. It reflects on how the proposed models address current sustainability and scalability challenges and outlines directions for future research, considering the broader implications of the presented contributions to lay the groundwork for advancing the field of sustainable transportation. 

The primary objective of this thesis has been to design, implement, and evaluate a ride-sharing service model that is both scalable and carbon-neutral, leveraging the synergies between AEVs and SECs. Recognising the urgent need for sustainable mobility solutions, this work has proposed models to formalise the partition of an AEVs fleet over the SECs of a city, as well as the allocation of TPs to AEVs, their routing and charging scheduling over a simulated time horizon. These models have integrated energy generation, allocation and re-routing constraints to maximise the number of TPs served. Together, these contributions present a robust architecture for operating future community-based ride-sharing services.

Chapter 3 introduced a reactive ride-sharing service formulated as a variant of the DVRPTW. A greedy decision-making strategy, executed within a reactive simulation, maximised the number of TPs served while maintaining system responsiveness. The accompanying instance generator extended both Google Hashcode and NYC taxi datasets, enabling rigorous testing under diverse conditions. Results showed quadratic complexity in the number of trips, solving instances with 1,000 trips in under a second and 10,000 trips in under two minutes. Applied to real NYC taxi traces, the service cut fleet size by 84\% at the cost of only a 21\% increase in distance travelled, underscoring its practical scalability and environmental benefit.

Building on this foundation, Chapter 4 framed the fleet-allocation task as a decentralised, iterative negotiation among independent SECs, each competing for AEV resources to satisfy local transport demand. A Multi-Objective, Multi-Agent reinforcement-learning approach—combining Deep Q-Learning with Graph Convolutional Networks—enabled SECs to learn efficient exchange policies using only local information. Experiments on adapted Google HashCode and NYC taxi instances demonstrated near-optimal allocations when inter-SEC connectivity was high, and consistent solution quality across thousands of trips and vehicles within seconds, confirming that decentralised control can match centralised performance while better accommodating dynamic urban conditions.

Chapter 5 completed the ride-sharing service with a reward-based charging scheduler that jointly optimised the number of TPs served and energy efficiency under constraints on energy use, vehicle distribution, route plans, and charging-station availability. A two-phase algorithm—vehicle routing followed by charging optimisation—prioritised charging at times and locations with favourable RES supply or lower grid load. Tests on parameterised benchmarks confirmed that the algorithm maintained high number of TPs served and charging efficiency when working with large fleets and request volumes, therefore reinforcing the viability of the novel carbon-neutral, community-based ride-sharing service.

Taken together, these three contributions form an integrated service in which (1) reactive eco-routing delivers instant trip-matching at scale, (2) decentralised fleet negotiation balances vehicles across heterogeneous zones, and (3) incentive-aligned charging minimises carbon footprint and grid stress. The combined system scales from neighbourhood to metropolitan settings, achieves substantial reductions in private-vehicle kilometres, and operates exclusively on RES electricity—all while meeting limited real-time computational budgets. As such, the PhD thesis offers a concrete blueprint for next-generation smart-city mobility that is simultaneously sustainable, resilient, and ready for real-world deployment.

Building on these promising results, several clear opportunities exist to expand this work even further. A logical next step is to pilot the models in different operational settings, such as integrating continuous data feeds—such as real-time traffic, dynamic electricity prices, and weather forecasts—to provide more fine-grained decision-making and adaptability. 

The proposed open-source, carbon-neutral ride-sharing model can be piloted directly within MTU’s existing Park and Ride scheme \cite{mtu2024parkride}. By conceptualising surrounding neighbourhoods as SEC nodes, commuting to college will reduce the carbon footprint. Practical integration could include linking the ride-sharing app directly with MTU’s student-card and staff-ID systems to provide access and reward mechanisms, leveraging real-time data streams from campus traffic sensors and renewable-energy generation dashboards to optimise vehicle charging schedules, and establishing geo-fenced pick-up zones to reduce congestion in high-traffic areas.

Such a pilot promises substantial societal benefits. First, affordability is enhanced, as ride-sharing significantly reduces commuting expenses, which is especially important for part-time students working to cover their living costs. In addition, sustainability is improved through increased use of shared EVs, reducing the university’s transport-related carbon footprint and aligning with the EU’s Climate Action Roadmap.

Furthermore, implementing this system within MTU's operational environment would generate detailed, real-world data—including traffic patterns, energy use, user preferences, and reward responsiveness—that could be directly utilised to refine the optimisation models introduced in Chapters 3 to 5. This feedback loop would test and validate the effectiveness of the proposed algorithms under realistic, dynamic conditions such as changing schedules, variable renewable energy availability, and surges in transportation demand during examination periods. Insights gained from this pilot could guide future improvements aimed at broader urban deployment.

As battery technologies, RES infrastructures, and communication systems evolve, adapting the models to leverage new advancements, such as V2G interactions and ultra-fast charging, will be essential for maintaining long-term relevance of the service. Additionally, a comprehensive lifecycle assessment of the ride-sharing service, including manufacturing, deployment, and decommissioning phases of AEVs and SECs, would provide a holistic view of the environmental benefits of the service and of its economic feasibility. 

By addressing these challenges and opportunities, the ride-sharing service developed in this PhD thesis can be further strengthened, contributing significantly to the realisation of sustainable, smart transportation ecosystems.

While the proposed models demonstrate promising results, several limitations inherent to the current implementation should be acknowledged. First, the eviction policy applied at charging stations adopts a simplified, system-wide perspective that prioritises vehicles with higher trip-serving potential. Although effective in maximising trip petitions served, this approach overlooks possible operational constraints, such as vehicle-specific priorities, or passenger convenience, which would likely influence real-world acceptance of such policies.

Second, the simulations assume a homogeneous AEV fleet in terms of battery capacity, charging profiles, and service capability. In practice, urban fleets are diverse, comprising vehicles with varying specifications, performance characteristics, and maintenance needs. Integrating heterogeneous vehicle types would add complexity but more accurately reflect real-world operational conditions.

Finally, the models simplify the calculation of travel distances and times, using idealised or grid-based approximations to maintain computational efficiency. While suitable for large-scale algorithmic testing, this abstraction does not account for urban traffic dynamics, road network irregularities, or real-time congestion patterns. Future extensions should incorporate high-resolution, real-world transport data to enhance result realism.

These assumptions, while necessary for the current stage of development, present clear avenues for refinement and further validation, particularly in preparation for deployment in complex, real-world mobility ecosystems.

