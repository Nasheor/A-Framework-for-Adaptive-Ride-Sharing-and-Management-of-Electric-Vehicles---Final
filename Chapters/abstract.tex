\begin{abstract}
Europe aims for carbon neutrality by 2050, with transportation currently amounting 20\% of
its total emissions. Motivated by the sustained mass transition to Autonomous Electric
Vehicles (AEVs) and by the distributed energy generation and storage of local Smart Energy
Communities (SECs), this thesis presents the design, implementation and evaluation of a
carbon-neutral, community-based, scalable ride-sharing service. Its models formalise the
partition of an AEVs fleet over the SECs of a city, as well as the allocation of trip petitions (TPs) to AEVs, their routing and charging scheduling over a simulated time horizon. These models integrate energy generation, allocation and re-routing constraints to maximise the number of TPs served. The solution approach to implement the models includes the
application of Greedy-based Heuristics and Metaheuristics, Reinforcement Learning and
Mixed Integer Programming techniques. A parameterised instance generator is developed,
aligning existing benchmarks (i.e. Google HashCode) and public datasets (i.e. NYC taxis) to
the proposed problem formulations and testing the service under various configurations.
The ride-sharing service is proven to scale well, when applied to very large instances,
providing fast and competitive results. Overall, this thesis contributes to green
transportation by providing a computational efficient, scalable, and environmentally
sustainable ride-sharing service.

% \keywords{ Trust}

\end{abstract}
