
List of publications (including any under review)
Abstract 
1.	Introduction  
a.	General overview of IoT & Trust
b.	Motivation – need for your work, work done and research challenges remaining and where your work lies – summary of chapter 2
c.	Research Questions 
i.	How to decentralize the trust management in IoT ecosystems?
ii.	How to leverage DLT for a decentralized IoT architecture? 
iii.	How to create an adaptive trust management model that can seamlessly consider future attack models?
iv.	How to perform an off-chain execution by maintaining the degree of decentralization and trust ? 
d.	Contribution
i.	DLT based trust framework for IoT ecosystems (overall contribution) – with the contributions below forming part of this overall framework
1.	Decentralized Trust Architecture for trust during design 
2.	Adaptive and Multi-Layer Trust Reputation Model for IoT systems
3.	Trusted work flows 
4.	DLT selection tool (we can see how this work pans out and introduce it as a minor contribution)
e.	Thesis Outline




2.	Trusted IoT Ecosystems
 (SoTA chapter, identify deficiencies and how your work addresses these – this should map directly to your contributions)
Expand on the introduction and discuss the general challenges for IoT, iot ecosystems (decentralised and distributed) in more detail and link these to trust issues where appropriate and where trust can be used to address these challenges. Talk about next generation internet where trust is a core element etc


Then discuss the need to look at trust from a design and operational perspective  

a.	On the design side discuss the existing architectures and limitations – as outlined below
b.	Operation: Trust mgt in Iot – as outlined below, you then need to discuss the need to move from centralised to decentralised trust and this should make the case for DLT and naturally lead into the next section 
c.	DLT for Decentralizing IoT Trust and the role of DLT in this and the motivation for your work 
d.	You need to discussion Trust calculation, reputation modelling etc and the role of ML in this and compare against SOTA 
e.	For the trusted ML work you need to introduce trusted workflows – this section should focus on trusted execution environments and how this relates to IoT ecosystems – as motivation for this and to link to the trusted discuss ML and the need for ML across the edge-cloud and the need for trusted execution environments/workflows to support this, this should  then motivate the need for the work with Sourabh – in terms of DLT a brief overview can be given in this section and further details can be included in an appendix as needed. 
f.	DLT selection tool – again we can see whether or not to include and digiblocks and smartQC deliverables can be leveraged for this
g.	From the SoTA limitiations defines the requirements for trusted iot ecostystems and Briefly introduction to your framework and its core objectives to address the SOTA limitations – cross layer DLT and ML based trust approach 
h.	Conclusion  



3.	A DLT Based Trust Framework 

a.	Architectural Design 
i.	Holistic Trust Management
ii.	Mapping to TIOATA Reference Architecture
b.	Trusted Life Cycle for IoT 
i.	Provisioning (ID Management)
ii.	Tracing (Interactions and Access Control)
iii.	Decommissioning (Trust Reputation)
c.	Machine Learning Based Trust Reputation
Dive deep in to the machine learning based cross layer trust reputation model that sits on the overall trust framework.
i.	Composition 
ii.	Layered Trust
iii.	Propagation
iv.	Aggregation

d.	Trusted Machine Learning Work Flow
Discuss the mechanism proposed to ensure the trusted machine learning model management. This will be link back to the management of the ML models used for trust reputation calculation. Furthermore, and additional use case on federated learning using the proposed method is also discussed here.
i.	Secure Model Provisioning
ii.	Secure Model Testing
e.	Use Cases
i.	Digiblocks
ii.	Federated Learning




4.	Framework Implementation
a.	Hyperledger Fabric Based Implementation
Fabric network design and implementation considering the IoT network requirements.  
Smart contract based implementation details of the Identity Management, Access Control and Trust Reputation, Deployment setups.

b.	Machine Learning Model Training 
Simulation setup and attack detection model training process. 
i.	Dataset
ii.	Attack Model
iii.	Model Training
c.	Off chain Execution 
i.	Intel SGX based implementation
5.	Use Case Evaluation
a.	Experiments and Results 
i.	Throughput 
ii.	Latency
iii.	Trust Convergence 
iv.	Energy Usage
v.	Availability
vi.	Time Complexity
b.	Comparison

6.	Conclusion and Future Work


